\documentclass[english,a4paper,twocolumn,oneside,article]{memoir}
\usepackage[utf8]{inputenc}
\usepackage[T1]{fontenc}
\usepackage{babel}
\usepackage{xspace}
\usepackage{microtype}
\usepackage[osf]{mathpazo}
\linespread{1.05}

\title{Bylaws for Fredagscaféen}
\author{}
\date{}

% Typografi: Marginer, spalter og lister
\setlrmarginsandblock{1.5cm}{*}{1}
\setulmarginsandblock{2.5cm}{*}{1.2}
\setlength\columnsep{3em}
\checkandfixthelayout[nearest]
\firmlists

% Typografi: \chapter er paragraf, \section er stykke. Typegrafien
% skal være særlig for at skabe overskuelighed og skabe plads.
\makechapterstyle{bylaws}{
  \def\chapterheadstart{\pagebreak[1]\vspace{\beforechapskip}}
  \def\printchaptername{}
  \def\printchaptertitle##1{\centering\chaptitlefont##1}
  \def\chapternamenum{}
  \def\printchapternum{\centerline{\chapnumfont\thechapter}}
}\chapterstyle{bylaws}
\setlength\beforechapskip{2\baselineskip plus 3pt minus 3pt}
\setlength\midchapskip{4pt plus 1pt minus 2pt}
\setlength\afterchapskip{1\baselineskip plus 1pt minus 2pt}
\renewcommand\chapnumfont{\huge\bfseries}
\renewcommand\chaptitlefont{\Large\bfseries}
\renewcommand{\thechapter}{\S\arabic{chapter}}
\renewcommand{\thesection}{Sec.~\arabic{section}}
\setsecheadstyle{\normalsize\bfseries}
\setlength\aftersecskip{0pt}
\setlength\beforesecskip{\baselineskip}

\makeatletter
\renewcommand{\p@section}{\thechapter~\expandafter\MakeLowercase}
\makeatother

\begin{document}

\maketitle
\newpage

Passed at the association's ordinary general assembly on Friday, February 28, 2025.

\chapter{Name and affiliation}

\section{} The name of the association is Fredagscaféen.

\section{} The association's home is at the Department of Computer Science at Aarhus University, Åbogade 34 8200 Aarhus N.

\chapter{Purpose}

\section{} Holding a Friday café for students and employees affiliated with the Department of Computer Science at 
Aarhus University.

\section{} Create a social forum where students and employees can meet in a cozy and informal setting.

\section{} Organizing social events for students and employees affiliated with the Department of Computer Science at 
Aarhus University.


\chapter{Members}

\section{} To be accepted as a member, one must be affiliated with IT City Katrinebjerg, either as a student or employee. 
The member has no legal obligations towards the association.

\section{} The board decides who will be admitted as a member based on the applications.

\section{} Members of the association are expected to comply with the rules of \ref{p:rules}.

\chapter{Rules for members}\label{p:rules}

\section{}\label{s:active} Members actively participate in conducting the association's events.

\section{} Neither the association's board nor other members may receive salaries or subsidies from the 
association.


\chapter{Resignation}

\section{} A qualified majority of 2/3 of the board may exclude a member of the association if
they do not comply with the rules specified in \ref{p:rules}, or if the majority of the board deems 
it necessary.

\section{} Any member may resign from the association with at least one month's notice to the current board.


\chapter{Board}

\section{} The association's board consists of:

\begin{itemize}
\item Chairman
\item Treasurer
\item 2--8 common board members
\item 1--2 board substitutes
\end{itemize}

\section{}\label{s:howvote} The board is elected at a general assembly. 
First the chairman is elected, then the treasurer, then the ordinary board members and finally the board substitutes. 
The chairman and treasurer must either be approved by a majority of the sitting board or have been members of the 
association for at least 1 year.

\section{} The board is responsible for the operation and administration of the association.

\section{} The chairman is responsible for the association's daily operations and for convening meetings 
with the board or all members of the association. The chairman is also responsible for convening the 
general assembly of the association.

\section{} The treasurer is responsible for preparing an annual accounts and presenting it at the ordinary general assembly.

\section{} The association is signed by the chairman and the treasurer.

\section{} Board substitutes are allowed to observe and assist with the operation of the board, but without voting rights.
If one or more board members, for a longer period of their board term, are prevented from performing their board duties,
then one or more board substitutes will be able to act as substitutes for the current board member(s) with full voting rights.

\chapter{General assembly}

\section{} The ordinary general assembly is held in the spring semester each year.

\section{} Notice of the general assembly must be sent to the association's members at least 2 weeks before the event. 
The agenda must be sent at least 1 week before the event.

\section{} Proposals submitted to the general assembly must reach the board no later than 1 week before 
the general assembly.

\section{} The general assembly has a quorum when 20\% of the association's members are present. 
Decisions are made by simple majority and members of the association have voting rights.

\section{} At ordinary general assemblies the following items must be addressed:

\begin{enumerate}
\item Formalities; election of conductor and rapporteur.
\item Report from the outgoing board.
\item Report on the association's accounts by the association's treasurer.
\item Processing of received proposals.
\item Evaluation of the association's work.
\item Election of the board based on \ref{s:howvote}.
\item Election of auditor.
\item AOB; Any other business.
\end{enumerate}

\section{} If at least one member present so wishes, the given vote shall be held in writing.

\section{} The board or a majority of the association's members may call an extraordinary general assembly. 
This must be done with at least 1 month's notice.


\chapter{Accounts}

\section{} The association's fiscal year runs from January 1 to December 31.

\section{} The auditor elected by the general assembly carries out a thorough critical audit and checks the accounting.


\chapter{Responsibility}

\section{} In the event of any deficit or similar in the association's activities, only the association, 
with its possible assets, is liable.


\chapter{Amendments}

\section{} Amendments to the bylaws must be considered at a general meeting, where decisions are made 
by a qualified majority of 2/3 of the members present.


\chapter{Dissolution of the association}

\section{} The association may be dissolved at a general meeting where 80\% of the association's members vote 
in favor of this proposal.

\section{}\label{s:disbandmentmoney} Upon dissolution of the association, all of the association's funds will 
be transferred to a student support fund.

\chapter{Passing and commencement}

\section{} These bylaws enter into force on the day after its passage.

\section{}\label{s:dateonbylaws} The date of the bylaws must be stated in the articles of association.

\section{} These bylaws replace all previous bylaws.

\end{document}
