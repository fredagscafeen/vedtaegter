\documentclass[danish,a4paper,twocolumn,oneside,article]{memoir}
\usepackage[utf8]{inputenc}
\usepackage[T1]{fontenc}
\usepackage{babel}
\usepackage{xspace}
\usepackage{microtype}
\usepackage[osf]{mathpazo}
\linespread{1.05}

\title{Vedtægter for Fredagscaféen}
\author{}
\date{}

% Typografi: Marginer, spalter og lister
\setlrmarginsandblock{1.5cm}{*}{1}
\setulmarginsandblock{2.5cm}{*}{1.2}
\setlength\columnsep{3em}
\checkandfixthelayout[nearest]
\firmlists

% Typografi: \chapter er paragraf, \section er stykke. Typegrafien
% skal være særlig for at skabe overskuelighed og skabe plads.
\makechapterstyle{bylaws}{
  \def\chapterheadstart{\pagebreak[1]\vspace{\beforechapskip}}
  \def\printchaptername{}
  \def\printchaptertitle##1{\centering\chaptitlefont##1}
  \def\chapternamenum{}
  \def\printchapternum{\centerline{\chapnumfont\thechapter}}
}\chapterstyle{bylaws}
\setlength\beforechapskip{2\baselineskip plus 3pt minus 3pt}
\setlength\midchapskip{4pt plus 1pt minus 2pt}
\setlength\afterchapskip{1\baselineskip plus 1pt minus 2pt}
\renewcommand\chapnumfont{\huge\bfseries}
\renewcommand\chaptitlefont{\Large\bfseries}
\renewcommand{\thechapter}{\S\arabic{chapter}}
\renewcommand{\thesection}{Stk.~\arabic{section}}
\setsecheadstyle{\normalsize\bfseries}
\setlength\aftersecskip{0pt}
\setlength\beforesecskip{\baselineskip}

\makeatletter
\renewcommand{\p@section}{\thechapter~\expandafter\MakeLowercase}
\makeatother

\begin{document}

\maketitle
\newpage

Vedtaget på foreningens ordinære generalforsamling fredag den 28.\ februar 2025.

\chapter{Navn og tilhørsforhold}

\section{} Foreningens navn er Fredagscaféen.

\section{} Foreningens hjemsted er Institut for Datalogi ved Aarhus Universitet, Åbogade 34 8200 Aarhus N.

\chapter{Formål}

\section{} Afholdelse af fredagscafé for studerende og ansatte med tilknytning til Institut for Datalogi
ved Aarhus Universitet.

\section{} Skabe et socialt forum hvor studerende og ansatte kan mødes i hyggelige og uformelle rammer.

\section{} Afholdelse af sociale arrangementer for studerende og ansatte med tilknytning til Institut for Datalogi
ved Aarhus Universitet.


\chapter{Medlemmer}

\section{} For at kunne blive optaget som medlem, skal man have tilhørsforhold til IT-byen Katrinebjerg,
enten som studerende eller ansat. Medlemmet har ingen retslige forpligtigelser over for foreningen.

\section{} Bestyrelsen beslutter ud fra ansøgningerne, hvem der optages som medlem.

\section{} Foreningens medlemmer forventes at overholde reglerne i \ref{p:rules}.

\chapter{Regler for medlemmer}\label{p:rules}

\section{}\label{s:active} Medlemmer deltager aktivt i afholdelse af foreningens arrangementer.

\section{} Hverken foreningens bestyrelse eller øvrige medlemmer kan få udbetalt løn eller
subsidier af foreningen.


\chapter{Udmeldelse}

\section{} Et kvalificeret flertal på 2/3 af bestyrelsen kan ekskludere et medlem af foreningen, hvis 
dette ikke overholder de i \ref{p:rules} angivne regler, eller hvis bestyrelsesflertallet vurderer det nødvendigt.

\section{} Ethvert medlem kan med minimum én måneds varsel, til den siddende bestyrelse, udtræde af foreningen.


\chapter{Bestyrelse}

\section{} Foreningens bestyrelse består af:

\begin{itemize}
\item Formand
\item Kasserer
\item 2--8 menige bestyrelsesmedlemmer
\item 1--2 bestyrelsessuppleanter
\end{itemize}

\section{}\label{s:howvote} Bestyrelsen vælges på en generalforsamling. Først vælges formanden, dernæst kassereren, så de menige
bestyrelsesmedlemmer og til sidst bestyrelsessuppleanter. Formanden og kasseren skal enten godkendes af et flertal
af den siddende bestyrelse eller have været medlemmer af foreningen i mindst 1 år.

\section{} Bestyrelsen har ansvaret for foreningens drift og administration.

\section{} Bestyrelsesformanden står for foreningens daglige drift og indkaldelse til møder med bestyrelsen
eller alle foreningens medlemmer. Bestyrelsesformanden har ligeledes ansvaret for at indkalde til
generalforsamling i foreningen.

\section{} Kassereren har ansvaret for udarbejdelsen af et årsregnskab samt fremlæggelse af dette til den
ordinære generalforsamling.

\section{} Foreningen tegnes af formanden og kassereren.

\section{} Bestyrelsessuppleanter har lov til at observere og hjælpe til med bestyrelsens drift, dog uden at have stemmeret.
Hvis en eller flere bestyrelsesmedlemmer, i en længere tid af sin bestyrelsesperiode, bliver forhindret i at udføre sin bestyrelsespost,
så vil en eller flere bestyrelsessuppleanter kunne agere som stedfortræder for de/-n gældende bestyrelsesmedlem/-er med fuld stemmeret.

\chapter{Generalforsamling}

\section{} Der afholdes ordinær generalforsamling i forårssemesteret hvert år.

\section{} Indkaldelse til generalforsamling skal ske mindst 2 uger før arrangementet via e-post til foreningens
medlemmer. Dagsordenen skal komme mindst 1 uge før arrangementet.

\section{} Indkomne forslag til generalforsamlingen skal være bestyrelsen i hænde senest 1 uge før generalforsamlingen.

\section{} Generalforsamlingen er beslutningsdygtig når 20\% af foreningens medlemmer er til stede. Der
besluttes med almindeligt flertal og stemmeret har medlemmer af foreningen.

\section{} På ordinære generalforsamlinger skal følgende punkter behandles:

\begin{enumerate}
\item Formalia; valg af dirigent og referent.
\item Beretning fra den afgående bestyrelse.
\item Beretning om foreningens regnskab af foreningens kasserer.
\item Behandling af indkomne forslag.
\item Evaluering af foreningens arbejde.
\item Valg af bestyrelse ud fra \ref{s:howvote}.
\item Valg af revisor.
\item Eventuelt.
\end{enumerate}

\section{} Hvis mindst ét fremmødt medlem ønsker det, skal den givne afstemning afholdes skriftligt.

\section{} Bestyrelsen eller et flertal blandt foreningens medlemmer kan indkalde til en ekstraordinær
generalforsamling. Dette skal ske med mindst 1 måneds varsel.


\chapter{Regnskab}

\section{} Foreningens regnskabsår løber fra 1. januar til 31. december.

\section{} Den af generalforsamlingen valgte revisor udøver en grundig kritisk revision og kontrollerer
regnskabsførelsen.


\chapter{Ansvar}

\section{} Ved eventuelt underskud eller lignende på foreningens aktiviteter, er det kun foreningen der, ved
sine eventuelle aktiver, hæfter.


\chapter{Vedtægtsændringer}

\section{} Ændringer af vedtægter skal behandles på en generalforsamling, hvor beslutninger tages af et kvalificeret flertal på 2/3 af fremmødte medlemmer.


\chapter{Foreningens opløsning}

\section{} Foreningen kan opløses på en generalforsamling, hvor 80\% af foreningens medlemmer stemmer
for dette forslag.

\section{}\label{s:disbandmentmoney} Ved opløsning af foreningen overføres alle foreningens midler til en studenterstøttende fond.

\chapter{Vedtagelse og ikrafttræden}

\section{} Disse vedtægter træder i kraft dagen efter vedtagelse.

\section{}\label{s:dateonbylaws} Datoen for vedtagelse skal påføres vedtægterne.

\section{} Disse vedtægter erstatter alle tidligere vedtægter.

\end{document}
